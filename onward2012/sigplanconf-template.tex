%-----------------------------------------------------------------------------
%
%               Template for sigplanconf LaTeX Class
%
% Name:         sigplanconf-template.tex
%
% Purpose:      A template for sigplanconf.cls, which is a LaTeX 2e class
%               file for SIGPLAN conference proceedings.
%
% Guide:        Refer to "Author's Guide to the ACM SIGPLAN Class,"
%               sigplanconf-guide.pdf
%
% Author:       Paul C. Anagnostopoulos
%               Windfall Software
%               978 371-2316
%               paul@windfall.com
%
% Created:      15 February 2005
%
%-----------------------------------------------------------------------------


\documentclass[preprint,10pt]{sigplanconf}

% The following \documentclass options may be useful:
%
% 10pt          To set in 10-point type instead of 9-point.
% 11pt          To set in 11-point type instead of 9-point.
% authoryear    To obtain author/year citation style instead of numeric.

\usepackage{amsmath}
\usepackage{enumerate}

\begin{document}

\conferenceinfo{WXYZ '05}{date, City.} 
\copyrightyear{2005} 
\copyrightdata{[to be supplied]} 

% \titlebanner{banner above paper title}        % These are ignored unless
% \preprintfooter{short description of paper}   % 'preprint' option specified.

\title{$title$}

\authorinfo{Toby Schachman}
           {New York University, Interactive Telecommunications Program}
           {tqs@alum.mit.edu}

\maketitle

\begin{abstract}
This paper seeks to broaden the view of what programming is, who programs, and how programming fits in to larger systems. With growing frequency, people are approaching programming from unlikely backgrounds such as the arts. Often these new programmers bring with them ways of working which are incompatible with mainstream programming practices, but which allow for new possibilities in programming interfaces. This paper makes suggestions for the design of these new programming interfaces. It presents as a case study and demonstration Recursive Drawing. Recursive Drawing is a reimplementation of the textual programming language Context Free as a graphical, directly manipulable interface. Instead of a compiler or interpreter, Recursive Drawing's programming interface is modeled as a constraint solver. This allows the programmer to modify the program's source code by manipulating the program's output. Additionally, the design of the interface focuses on program transformation, rather than program construction.
\end{abstract}

\category{CR-number}{subcategory}{third-level}

\terms
term1, term2

\keywords
keyword1, keyword2

$body$

\appendix
\section{Appendix Title}

This is the text of the appendix, if you need one.

\acks

Acknowledgments, if needed.

% We recommend abbrvnat bibliography style.

\bibliographystyle{abbrvnat}

% The bibliography should be embedded for final submission.

\begin{thebibliography}{}
\softraggedright

\bibitem[Smith et~al.(2009)Smith, Jones]{smith02}
P. Q. Smith, and X. Y. Jones. ...reference text...

\end{thebibliography}

\end{document}
